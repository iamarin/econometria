\documentclass[a4paper,11pt]{book}
\usepackage{enumerate}
\usepackage{amssymb, amsmath, amsbsy}
\usepackage{upgreek}
\usepackage{mathdots}
\usepackage{mathrsfs} 
\usepackage{stackrel} 
\begin{document}
\title{Apuntes de econometría}
\author{Iván Marín}
\date{Enero, 2011}
\maketitle
\chapter{Introducción}
El presente documento es el resultado de los esfuerzos para proporcionar una guía que sirva de instrumento revisorio de los elementos expuestos por Wooldrige.

\chapter{Modelo de regresión simple}
\chapter{Modelo de regresión simple}
\section{a ver}
\chapter{Modelo de regresión múltiple: estimación}
\chapter{Modelo de regresión múltiple: inferencia}
\chapter{Modelo de regresión múltiple: MCO asintóticos}
\chapter{Modelo de regresión múltiple: temas adicionales}
\chapter{Modelo de regresión múltiple con información cualitativa}
\chapter{Modelo de regresión con ST}
\chapter{Aspectos MCO adicionales en ST}
\chapter{Correlación serial y heterocedasticidad en ST}
\chapter{Combinación de CT en ST: DP}
\chapter{Métodos avazados para DP}
\chapter{VI y MC2E}
\chapter{Modelos de ecuaciones simultáneas}
En el capítulo anterior se mostró que el método VI resulve dos problemas:

\begin{itemize}
	\item Variables omitidas
	\item Error de medición
\end{itemize}
El promblema que enfrentamos ahora es el de la \textbf{simultaneidad}, esto implica que dos o más variables se determinan conjuntamente.
Este apartado considera las siguientes secciones:
\begin{itemize}
	\item Naturaleza y alcance del MES
	\item ¿MCO para una ecuación del MES?: SESGO e INCONSISTENCIA
	\item Identificación y estimación con dos ecuaciones
	\item Identificación y estimación con más de dos ecuaciones
	\item Ecuaciones simultáneas con ST
	\item Ecuaciones simultáneas con DP
\end{itemize}

\section{Naturaleza y alcance del MES}

Lo más importante cuando trabajamos con un Modelo de Ecuaciones Simultáneas(MES) es recordar que cada ecuación del sistema tiene una interpretación \textit{ceteris paribus}. Por ejemplo, consideremos una función simple de oferta de mano de obra:
\begin{equation}\label{eq:mh}
h_{s} = \alpha_{1}w + \beta_{1}z_{1} + u_{1}
\end{equation}

\chapter{Modelos de variable dependiente limitadas y correciones de selección muestral}
\chapter{Temas avanzados de ST}

\end{document}
